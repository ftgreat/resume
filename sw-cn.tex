\documentclass[11pt,a4paper]{moderncv}

% moderncv themes
%\moderncvtheme[blue]{casual}                 % optional argument are 'blue' (default), 'orange', 'red', 'green', 'grey' and 'roman' (for roman fonts, instead of sans serif fonts)
\moderncvtheme[green]{classic}                % idem
\usepackage{xunicode, xltxtra}
\XeTeXlinebreaklocale "zh"
\widowpenalty=10000

%\setmainfont[Mapping=tex-text]{文泉驿正黑}

% character encoding
%\usepackage[utf8]{inputenc}                   % replace by the encoding you are using
\usepackage{CJKutf8}
  
% adjust the page margins ,设置页面边距
\usepackage[scale=0.8]{geometry}
\recomputelengths                             % required when changes are made to page layout lengths
\setmainfont[Mapping=tex-text]{Hiragino Sans GB}
\setsansfont[Mapping=tex-text]{Hiragino Sans GB}
\CJKtilde

% personal data
\firstname{宋伟}
\familyname{}
\title{}               % optional, remove the line if not wanted

\mobile{18810332274}                    % optional, remove the line if not wanted
\email{ningyuwhut@gmail.com}                      % optional, remove the line if not wanted
%% \quote{\small{``Do what you fear, and the death of fear is certain.''\\-- Anthony Robbins}}

\nopagenumbers{}

\begin{document}

\maketitle

\section{教育}
\cventry{2011至2014}{硕士}{北京航空航天大学计算机学院,导师:郎波}{}{}{}
\cventry{2007--2011}{本科}{武汉理工大学计算机学院}{}{}{}  

\section{工作经历}
\cventry{2014.04-至今}{搜狗营销事业部pc搜索广告研发组}{副研究员}{客户优化相关的算法研究}{}{}
%\cventry{2011}{EMC 上海研究院}{}{}{}{关于共享虚拟化平台上 I/O 旁路攻击的研究。}

%\section{社区}
%\cventry{Blog}{\url{blog.yxwang.me}}{技术博客}{}{}{}
%\cventry{StackOverflow}{\url{stackoverflow.com/users/111896/zellux}}{总排名前 4\%}{}{}{}
%\cventry{GitHub}{\url{github.com/zellux}}{参与和创建过多个开源项目}{}{}{}

\section{项目经历}
\renewcommand{\baselinestretch}{1.2}
\cventry{2015.06--2015.09}
{零星广告识别算法优化}
{python,shell}
{}{}
{优化已有的零星广告流程,一期主要优化时间和空间性能,流程时间由16h降低至8h。二期优化算法的准确率和召回率,在尝试过gbdt和两种不平衡算法后,算法的召回率在60\%,准确率在30\%。}

\vspace*{0.2\baselineskip}
\cventry{2015.01--2015.06}
{印第安纳琼斯数据需求平台}
{python}
{}{}
{自动化完成产品经理的数据需求。前端将请求解析成配置文件,后端获取后在hadoop集群上处理并生成结果文件,最后在本地合并后发送给收件人。平台可以承担组内产品经理90\%的数据需求。}

\vspace*{0.2\baselineskip}
\cventry{2014.04--2014.09}
{创意推荐流程的优化}
{python,shell}
{}{}
{接手创意推荐系统,优化该系统的效率,通过文本去重将创意库缩小30\%。并在该系统基础上实现一个简化版本加入到客户优化工具箱中。}

\vspace*{0.2\baselineskip}
\cventry{2013--2014}
{基于多模态PLSA的图像检索算法的研究与实现}
{Java,JSP}
{毕设}{}
{将传统的PLSA算法从单模态扩展到多模态,即同时通过文本和图像内容(两种模态)进行检索。实验结果表明,多模态的PLSA算法 可以有效提高检索准确度。}

%\vspace*{0.2\baselineskip}
%\cventry{2010}
%{ForceField}
%{Python, ANTLR}
%{独立项目}{}
%{设计了~ForceField~函数式语言,并借助~Python~和~ANTLR~为它实现了一个简单的解释器。为了更好的调试~ForceField~脚本,我们还开发了一个在线的调试工具,开发者可以利用它在网页上进行单步跟踪或者查看局部变量等操作。这个项目让我们获得了~2011~年盛大``校园牛人''创新技术大赛一等奖。}
%
%\vspace*{0.2\baselineskip}
%\cventry{2010}
%{AppFlight for Android}
%{Android 应用}
%{创业项目}{}
%{根据用户在网页~CMS~上定制的内容,生成一个显示这些内容的~Android~应用。}
%
%\vspace*{0.2\baselineskip}
%\cventry{2009}
%{Nexus}
%{C, 汇编}
%{独立项目}{}
%{基于~MIT 6.828~课程的~lab,实现了一个类似~exokernel~的微内核,包括多任务调度、文件系统、网络传输等基本功能。}
%
%\vspace*{0.2\baselineskip}
%\cventry{2009}
%{阅读进度记录}
%{Python, Django, Google App Engine}
%{独立项目}{}
%{记录用户阅读进度的~Web~应用,可以与豆瓣同步阅读列表。}
%
%\vspace*{0.2\baselineskip}
%\cventry{2008}
%{Koprulu}
%{C, 汇编}
%{课程项目}{}
%{
%基于~CMU~15-410~课程~lab~,实现了一个类似~Linux~的支持多任务调度的操作系统。
%}
%
%\vspace*{0.2\baselineskip}
%\cventry{2008}
%{绩点计算器}
%{Python, wxWidgets}
%{独立项目}{}
%{一个抓取学校查分系统网页,解析课程成绩数据,并计算各类绩点的应用,发布的第一年里下载量超过了~6000~份。}

%% \cventry{2010至今}{Danimos}{C, 汇编}{}{}{通过虚拟化技术保护在不可信的操作系统中运行的浏览器}
%% \cventry{2008}{GPGPU平台和优化}{C++, CTM IL}{}{}{高性能通用显卡计算与平台特性的研究,曦源研究项目}

%% \subsection{课程项目}
%% \cventry{2008}{分布式IRC服务器}{C}{}{}{}
%% \cventry{2008}{Tiger语言编译器}{C, Flex, Bison}{}{}{包括词法、语法分析,后端代码生成等}

%% \subsection{其他项目}
%% \cventry{2008}{Koprulu}{C, 汇编}{}{}{简单的类Linux多任务操作系统内核,基于CMU 15-410课程的lab}
%% \cventry{2008}{长征社区捐助管理系统}{Java, JSP}{}{}{基于Web的捐助管理系统}
%% \cventry{2007}{协作式多客户端在线绘图系统}{Java, RMI}{}{}{}
%% \cventry{2007}{复旦大学火车订票系统}{Java, JSP}{}{}{大一初学Web后接的外包}

\renewcommand{\baselinestretch}{1.0}

%\section{发表论文}
%\cventry{2012}
%{\textbf{Yuanxuan Wang}\textnormal{, Jincheng Han, Haibo Chen, Binyu Zang}}
%{ReDroid: Fast Malware Recovery on Smartphones with Intertwined Replay}{In IPADS Technique Report}
%{}{}{}
%
%\cventry{2011}
%{\textnormal{Xiang Song, Haibo Chen, Rong Chen, }Yuanxuan Wang\textnormal{, Binyu Zang}}
%{A Case for Scaling Applications to Many-core Platforms with OS Clustering}{In 2011 ACM SIGOPS European Conference on Computer Systems (Eurosys'2011) }
%{}{}{}

%% \section{助教经验}
%% \cventry{2011}{计算机系统工程}{}{}{}{}
%% \cventry{2009,2010}{操作系统}{}{}{}{}

%\section{奖项}
%\cventry{2012}{摩根士丹利暑期实习生团队项目最具创意奖}{}{}{}{}
%\cventry{2010}{SAP~大学联盟仪表盘设计大赛一等奖}{}{}{}{}
%\cventry{2010}{盛大``校园牛人''创新技术大赛一等奖}{}{}{}{}
%\cventry{2010}{研究生入学一等奖学金}{}{}{}{}
%%% \cventry{2009}{人民奖学金三等奖}{}{}{}{}
%%% \cventry{2007, 2008}{人民奖学金二等奖}{}{}{}{}
%\cventry{2007}{张江杯~DotA~比赛冠军}{}{}{}{}
%\cventry{2006, 2008}{复旦大学电子竞技大赛星际项目亚军}{}{}{}{}
%%% \cventry{2006}{首届``华夏通信杯''程序设计赛三等奖}{}{}{}{}
%\cventry{2005, 2004}{全国青少年信息学(计算机)奥林匹克分区联赛一等奖}{}{}{}{}

\section{技能}
\cventry{语言}{熟练使用python,shell,awk,熟悉java和c++}{}{}{}{}
\cventry{机器学习}{熟悉常用的机器学习算法(LR、SVM、GBDT等tree-based model、主题模型等算法,实现过一些常用的机器学习算法。}{}{}{}{}
%% \cventry{应用开发}{Ruby on Rails, iOS}{}{}{}{}
%\cventry{游戏}{星际争霸2(国服、东南亚服大师组)}{}{}{}{}
\cventry{英语}{CET-6 575 }{流利的英语读写能力}{}{}{}

% \cvline{Photography}{\small Digital photography is my newest hobby.}

\closesection{}                   % needed to renewcommands
\renewcommand{\listitemsymbol}{-} % change the symbol for lists

\end{document}
